\documentclass{ctexart}	%ctexbook,ctexrep
%导言区
\usepackage{ctex}
\usepackage{amsmath}
%使用newcommand制定命令,第一个参数命令的名字,第二个参数命令的内容。
\newcommand{\adots}{\mathinner{\mkern2mu%
		\raisebox{0.1em}{.}\mkern2mu\raisebox{0.4em}{.}%
		\mkern2mu\raisebox{0.7em{.}\mkern1mu}}}

%正文区
\begin{document}
	%矩阵环境中,用&分隔列,用\\分隔行
	\[
	\begin{matrix}
		0&1 \\
		1&1
	\end{matrix}
	%pmatrix(小括号)
	\begin{pmatrix}
		0&1\\
		1&0
	\end{pmatrix}
	%bmatrix(中括号)
	\begin{bmatrix}
		1&2\\
		4&3
	\end{bmatrix}
	%Bmatrix(大括号)
	\begin{Bmatrix}
		1&4\\
		2&8
	\end{Bmatrix}
	%vmatrix(单竖线)
	\begin{vmatrix}
		2&4\\
		6&8
	\end{vmatrix}
	%Vmatrix(双竖线)
	\begin{Vmatrix}
		5&10\\
		20&10
	\end{Vmatrix}
	\]
	
	%矩阵与另外的方法联立
	\[
	A=\begin{pmatrix}
		a_{11}^2&a_{12}^2&a_{452}^{567}\\
		0&a_{22}&a_{23}\\
		0&0&a_{33}
	\end{pmatrix}
	\]
	%常用省略号:\dots(横)、\vdots(竖)、\ddots(斜)
	\[
	A=\begin{bmatrix}
	a_{11}&\dots&a_{1n}\\
	&\ddots&\vdots\\
	0&&a_{nn}
	\end{bmatrix}_{n \times n}
	\]
	
	%分块矩阵(矩阵嵌套)
	\[
	\begin{pmatrix}
		\begin{pmatrix}
			1&0\\0&1
		\end{pmatrix} &
		\begin{pmatrix}
			5&6\\7&8
		\end{pmatrix}  \\
		\begin{pmatrix}
			7&8\\9&10
		\end{pmatrix} &
		\begin{pmatrix}
			11&12\\13&14
		\end{pmatrix}
	\end{pmatrix}_{n \times n}
	\]
	\[
	\begin{pmatrix}
		\begin{pmatrix}
		1&0\\0&1
		\end{pmatrix} &
		\text{\Large 0} \\	%\text可以独占一个分块矩阵单元
		\text{\Large 0} &
		\begin{pmatrix}
		11&12\\13&14
		\end{pmatrix}
	\end{pmatrix}_{n \times n}
	\]
	%三角矩阵
	\[
	\begin{pmatrix}
	a_{11}&a_{12}&\cdots&a_{1n}\\
	&a_{22}&\cdots&a_{2n}\\
	&	&\ddots&\vdots\\
	\multicolumn{2}{c}{\raisebox{1.3ex}[0pt]{\Huge 0}}%用multicolumn表示三角部分
	&                    &a_{nn}
	\end{pmatrix}
	\]
	%跨列的省略好:\hdotsfor{列数}
	\[
		\begin{pmatrix}
			1&\frac{1}{2}&\dots&\frac{1}{n}\\
			\hdotsfor{4}\\%用关键字:hdotsfor
			m&\frac{m}{2}&\dots&\frac{m}{n}
		\end{pmatrix}
	\]
	%行内小矩阵(smallmatrix)环境
	复数$z=(x,y)$也可用矩阵
	\begin{math}
		\left(%需要手动加上左括号
		\begin{smallmatrix}
		x&-y\\y&x
		\end{smallmatrix}
		\right)%需要手动加上右括号
	\end{math}
	%array环境(类似与表格环境tabular)
	\[
	\begin{array}{r|r}%rlc和竖线制定位置格式
	\frac{1}{2} & 0\\
	\hline%产生横线
	0&-\frac{a}{bc}\\
	\end{array}
	\]
	
	
	
	
\end{document}
