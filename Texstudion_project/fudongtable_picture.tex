\documentclass{ctexart}%ctexbook,ctexrep,详见 texdoc booktab,texdoc longtab,texdoc tabu
\usepackage{ctex}
\usepackage{graphicx}
\graphicspath{{figure/}}

%正文
\begin{document}
	%浮动体
	%实现灵活分页(避免无法分割的内容产生的页面留白)
	%给图表添加标题
	%交叉引用
	
	%figure环境(table环境与之类似)
	%\begin{figure}
	%	content...
	%\end{figure}
	
	%<允许位置>参数(默认tbp)
	%h,此处(here)-代码所在的上下文位置
	%t,页顶(top)-代码所在的页面或者之后页面的顶部
	%b,页底
	%p,独立一页
	%标题控制(caption,bicaption等宏包设置)
	%排绕(picinpar、wrapfig等宏包)
	%并排与子图表(subcaption、subfig、floatrow等宏包)
	\LaTeX{}中的插图,\TeX 系统的吉祥物---小狮子见图\ref{fig}:	%\ref{label}引用标签
	
	\begin{figure}[htbp]%可选参数指定排版位置
		
		\centering	%居中
		\includegraphics[scale=0.3]{vgg16.pdf}
		\caption{\TeX 系统的吉祥物}\label{fig}	%设置标题,\label{key}设置标签
		
	\end{figure}
	
	如表\ref{key}所示:
	\begin{table}[h]
		\centering 
		\caption{考试成绩}\label{key}
		
		\begin{tabular}{l|| c |c| p{1.5cm}| r}	
			%生成表格tabular。5列表格,
			%l左对齐,c居中对齐,r右对齐。
			%竖线符号表示竖线,两个竖线表示表格双竖线。
			%用p{<宽>}生成指定宽度的表格列,当内容超过时自动产生换行。
			\hline	%产生表格横线
			姓名&语文&数学&外语&备注\\
			\hline \hline	%两个hline产生双横线
			张三&87&100&93&优秀\\
			\hline 
			
		\end{tabular}
	\end{table}

\end{document}