\documentclass{article}
\usepackage{ctex}
\usepackage{graphicx}
\usepackage{pythonhighlight}
\graphicspath{{figure/}}

%\usepackage{listings}
%\input{./setting/en_preamble.tex}
%\input{./setting/xecjk_preamble.tex}
%\usepackage{xcolor}

%\lstset{
%	basicstyle          =   \sffamily,          % 基本代码风格
%	keywordstyle        =   \bfseries,          % 关键字风格
%	commentstyle        =   \rmfamily\itshape,  % 注释的风格,斜体
%	stringstyle         =   \ttfamily,  % 字符串风格
%	flexiblecolumns,                % 别问为什么,加上这个
%	numbers             =   left,   % 行号的位置在左边
%	showspaces          =   false,  % 是否显示空格,显示了有点乱,所以不现实了
%	numberstyle         =   \zihao{-5}\ttfamily,    % 行号的样式,小五号,tt等宽字体
%	showstringspaces    =   false,
%	captionpos          =   t,      % 这段代码的名字所呈现的位置,t指的是top上面
%	frame               =   lrtb,   % 显示边框
%}
%
%\lstdefinestyle{Python}{
%	language        =   Python, % 语言选Python
%	basicstyle      =   \zihao{-5}\ttfamily,
%	numberstyle     =   \zihao{-5}\ttfamily,
%	keywordstyle    =   \color{blue},
%	keywordstyle    =   [2] \color{teal},
%	stringstyle     =   \color{magenta},
%	commentstyle    =   \color{red}\ttfamily,
%	breaklines      =   true,   % 自动换行,建议不要写太长的行
%	columns         =   fixed,  % 如果不加这一句,字间距就不固定,很丑,必须加
%	basewidth       =   0.5em,
%}

%\usepackage{xcolor}
%\lstset{
%	numbers=left, 
%	numberstyle= \tiny, 
%	keywordstyle= \color{ blue!70},
%	commentstyle= \color{red!50!green!50!blue!50}, 
%	frame=shadowbox, % 阴影效果
%	rulesepcolor= \color{ red!20!green!20!blue!20} ,
%	escapeinside=``, % 英文分号中可写入中文
%	xleftmargin=2em,xrightmargin=2em, aboveskip=1em,
%	framexleftmargin=2em
%}




\title{周报}
\author{Shixiang Yan}
\date{\today}

\begin{document}
	\maketitle
	\tableofcontents
	\section{关于卫星天线仿真}
	我认为应该先从软件着手,对于STK我发现了有Matlab,C,C++,C\#,Java的接口。综合考虑开发难度,开发工作量以及业务需求,Matlab是最可靠的接口开发工具。本周我学习了MATLAB的相关知识,对其工作流程有了大致的了解。
	
	对于总体设计,我的想法是先在一台计算机上把所需功能跑出来,然后再考虑分布式到多台电脑上并行计算。下周我准备利用Matlab开发一个简单的可以与STK进行交互可视化程序。
	
	以下是我近期对Matlab学习所书写的部分代码:
	\subsection{结构}
	\subsubsection{顺序}
	\subsubsection{选择}
	\begin{python}
		%switch
		input_num=1;
		switch input_num
		case -1
		disp('negative 1');
		case 0
		disp('zero');
		case 1
		disp('positive 1');
		otherwise
		disp('other value');
		end        
	\end{python}
	\subsubsection{循环}
	\begin{python}
		%while循环
		n=1;
		while prod(1:n)<1e100 %Á¬³Ë
		n=n+1;
		end
		disp(n)
		%for循环
		for n=1:10
		a(n)=2^n;
		end
		disp(a)
		%%
		tic
		for ii=1:2000
		for jj=1:2000
		A(ii,jj)=ii+jj;
		end
		end
		toc
		%%
		tic
		A=zeros(2000,2000);
		for ii=1:size(A,1)
		for jj=1:size(A,2)
		A(ii,jj)=ii+jj;
		end
		end
		toc
	\end{python}
	\subsection{matlab声明函数}
	\begin{python}
		function result = fun2( a,b )
		%UNTITLED2 此处显示有关此函数的摘要
		%   此处显示详细说明
		s=0;
		for i=a:b
		s=s+i;
		end
		result=s;
		end
	\end{python}
	\subsection{画图}
	\subsubsection{正弦曲线}
	\begin{python}
		hold on
		plot(cos(0:pi/20:pi*2));
		plot(sin(0:pi/20:pi*2),'xg:');
		hold off
	\end{python}

	如图\ref{fig1}:
	
	\begin{figure}[h]
		\centering
		\includegraphics{plot1.ps}
		\caption{正弦曲线}\label{fig1}
	\end{figure}
	\section{关于Pytorch模块的学习}
	\subsection{numpy和pytorch}
	以下是近期我对Pytorch进行学习所书写的部分代码。
	\begin{python}
		#官网说明书:https://pytorch.org/docs/stable/torch.html
		import torch
		import numpy as np
		
		np_data = np.arange(6).reshape((2, 3))
		torch_data = torch.from_numpy(np_data)
		tensor2array = torch_data.numpy()
		print(
		'\nnumpy array:', np_data,          # [[0 1 2], [3 4 5]]
		'\ntorch tensor:', torch_data,      #  0  1  2 \n 3  4  5    [torch.LongTensor of size 2x3]
		'\ntensor to array:', tensor2array, # [[0 1 2], [3 4 5]]
		)
		
		# abs 绝对值计算
		data = [-1, -2, 1, 2]
		tensor = torch.FloatTensor(data)  # 转换成32位浮点 tensor
		print(
		'\nabs',
		'\nnumpy: ', np.abs(data),          # [1 2 1 2]
		'\ntorch: ', torch.abs(tensor)      # [1 2 1 2]
		)
		
		# sin   三角函数 sin
		print(
		'\nsin',
		'\nnumpy: ', np.sin(data),      # [-0.84147098 -0.90929743  0.84147098  0.90929743]
		'\ntorch: ', torch.sin(tensor)  # [-0.8415 -0.9093  0.8415  0.9093]
		)
		
		# mean  均值
		print(
		'\nmean',
		'\nnumpy: ', np.mean(data),         # 0.0
		'\ntorch: ', torch.mean(tensor)     # 0.0
		)
		
		# matrix multiplication 矩阵点乘
		data = [[1,2], [3,4]]
		tensor = torch.FloatTensor(data)  # 转换成32位浮点 tensor
		# correct method
		print(
		'\nmatrix multiplication (matmul)',
		'\nnumpy: ', np.matmul(data, data),     # [[7, 10], [15, 22]]
		'\ntorch: ', torch.mm(tensor, tensor)   # [[7, 10], [15, 22]]
		)
		
		# !!!!  下面是错误的方法 !!!!
		data = np.array(data)
		print(
		'\nmatrix multiplication (dot)',
		'\nnumpy: ', data.dot(data),        # [[7, 10], [15, 22]] 在numpy 中可行
		# '\ntorch: ', tensor.dot(tensor)     # torch 会转换成 [1,2,3,4].dot([1,2,3,4) = 30.0
		#'tensor.dot(tensor)',      torch 会转换成 [1,2,3,4].dot([1,2,3,4) = 30.0
		# 变为
		# '\ntorch:',torch.dot(tensor.dot(tensor))
		)
		\subsection{有关variable}
		\begin{python}
			import torch
			from torch.autograd import Variable # torch 中 Variable 模块
			
			# 先生鸡蛋
			tensor = torch.FloatTensor([[1,2],[3,4]])
			# 把鸡蛋放到篮子里, requires_grad是参不参与误差反向传播, 要不要计算梯度
			variable = Variable(tensor, requires_grad=True)
			
			print(tensor)
			"""
			1  2
			3  4
			[torch.FloatTensor of size 2x2]
			"""
			
			print(variable)
			"""
			Variable containing:
			1  2
			3  4
			[torch.FloatTensor of size 2x2]
			"""
			
			t_out = torch.mean(tensor*tensor)       # x^2
			v_out = torch.mean(variable*variable)   # x^2
			print(t_out)
			print(v_out)    # 7.5
			
			v_out.backward()    # 模拟 v_out 的误差反向传递,反向传递求梯度。
			
			# 下面两步看不懂没关系, 只要知道 Variable 是计算图的一部分, 可以用来传递误差就好.
			# v_out = 1/4 * sum(variable*variable) 这是计算图中的 v_out 计算步骤
			# 针对于 v_out 的梯度就是, d(v_out)/d(variable) = 1/4*2*variable = variable/2
			
			print(variable.grad)    # 初始 Variable 的梯度
			'''
			0.5000  1.0000
			1.5000  2.0000
			'''
			
			print(variable)     #  Variable 形式
			"""
			Variable containing:
			1  2
			3  4
			[torch.FloatTensor of size 2x2]
			"""
			
			print(variable.data)    # tensor 形式
			"""
			1  2
			3  4
			[torch.FloatTensor of size 2x2]
			"""
			
			print(variable.data.numpy())    # numpy 形式
			"""
			[[ 1.  2.]
			[ 3.  4.]]
			"""
		\end{python}
	\subsection{激活函数}
	\begin{python}
		import torch
		import torch.nn.functional as F     # 激励函数都在这
		from torch.autograd import Variable
		
		# 做一些假数据来观看图像
		x = torch.linspace(-5, 5, 200)  # x data (tensor), shape=(100, 1)
		x = Variable(x)
		
		x_np = x.data.numpy()   # 换成 numpy array, 出图时用,张量转numpy
		
		# 几种常用的 激励函数
		y_relu = torch.relu(x).data.numpy()
		y_sigmoid = torch.sigmoid(x).data.numpy()
		y_tanh = torch.tanh(x).data.numpy()
		y_softplus = F.softplus(x).data.numpy()
		# y_softmax = F.softmax(x)  softmax 比较特殊, 不能直接显示, 不过他是关于概率的, 用于分类
		
		import matplotlib.pyplot as plt  # python 的可视化模块, 我有教程 (https://morvanzhou.github.io/tutorials/data-manipulation/plt/)
		
		plt.figure(1, figsize=(8, 6))
		plt.subplot(221)
		plt.plot(x_np, y_relu, c='red', label='relu')
		plt.ylim((-1, 5))
		plt.legend(loc='best')
		
		plt.subplot(222)
		plt.plot(x_np, y_sigmoid, c='red', label='sigmoid')
		plt.ylim((-0.2, 1.2))
		plt.legend(loc='best')
		
		plt.subplot(223)
		plt.plot(x_np, y_tanh, c='red', label='tanh')
		plt.ylim((-1.2, 1.2))
		plt.legend(loc='best')
		
		plt.subplot(224)
		plt.plot(x_np, y_softplus, c='red', label='softplus')
		plt.ylim((-0.2, 6))
		plt.legend(loc='best')
		
		plt.show()
	\end{python}
	\subsection{Regressin}
	\begin{python}
		import torch
		import matplotlib.pyplot as plt
		
		x = torch.unsqueeze(torch.linspace(-1, 1, 100), dim=1)  \# x data (tensor), shape=(100, 1),数据维度进行扩充
		y = x.pow(2) + 0.2*torch.rand(x.size())                 \# noisy y data (tensor), shape=(100, 1),生成与x形状相同的y
		
		\# 画图
		plt.scatter(x.data.numpy(), y.data.numpy())
		plt.show()
		
		class Net(torch.nn.Module):  \# 继承 torch 的 Module
		def \_\_init\_\_(self, n\_feature, n\_hidden, n\_output):
		super(Net, self).\_\_init\_\_()     \# 继承 \_\_init\_\_ 功能
		\# 定义每层用什么样的形式
		self.hidden = torch.nn.Linear(n\_feature, n\_hidden)   \# 隐藏层线性输出,nn.Linear表示y=wx+b
		self.predict = torch.nn.Linear(n\_hidden, n\_output)   \# 输出层线性输出
		
		def forward(self, x):   \# 这同时也是 Module 中的 forward 功能
		\# 正向传播输入值, 神经网络分析出输出值
		x = torch.relu(self.hidden(x))      \# 激励函数(隐藏层的线性值)
		x = self.predict(x)             \# 输出值
		return x
		
		net = Net(n\_feature=1, n\_hidden=10, n\_output=1)
		
		print(net)  \# net 的结构
		"""
		Net (
		(hidden): Linear (1 -> 10)
		(predict): Linear (10 -> 1)
		)
		"""
		
		\# optimizer 是训练的工具
		optimizer = torch.optim.SGD(net.parameters(), lr=0.2)  \# 传入 net 的所有参数, 学习率
		loss\_func = torch.nn.MSELoss()      \# 预测值和真实值的误差计算公式 (均方差)
		
		for t in range(100):
		prediction = net(x)     \# 喂给 net 训练数据 x, 输出预测值
		
		loss = loss\_func(prediction, y)     \# 计算两者的误差
		
		optimizer.zero\_grad()   \# 清空上一步的残余更新参数值
		loss.backward()         \# 误差反向传播, 计算参数更新值
		optimizer.step()        \# 将参数更新值施加到 net 的 parameters 上
		
		\# import matplotlib.pyplot as plt
		
		plt.ion()   \# 画图
		plt.show()
		
		for t in range(50):
		
		prediction = net(x)  \# 喂给 net 训练数据 x, 输出预测值
		
		loss = loss\_func(prediction, y)  \# 计算两者的误差
		
		optimizer.zero\_grad()  \# 清空上一步的残余更新参数值
		loss.backward()
		optimizer.step()
		
		\# 接着上面来
		if t \% 10 == 0:
		\# plot and show learning process
		plt.cla()
		plt.scatter(x.data.numpy(), y.data.numpy())
		plt.plot(x.data.numpy(), prediction.data.numpy(), 'r-', lw=5)
		plt.text(0.5, 0, 'Loss=\%.4f' \% loss.data.numpy(), fontdict=\{'size': 20, 'color':  'red'\})
		plt.pause(0.1)
	\end{python}
	\subsection{classification}
		\begin{python}
			import torch
			import matplotlib.pyplot as plt
			
			# 假数据
			n_data = torch.ones(100, 2)         # 数据的基本形态
			
			x0 = torch.normal(2*n_data, 1)      # 类型0 x data (tensor), shape=(100, 2)
			help(torch.normal)
			print(x0)
			y0 = torch.zeros(100)               # 类型0 y data (tensor), shape=(100, )
			x1 = torch.normal(-2*n_data, 1)     # 类型1 x data (tensor), shape=(100, 1)
			y1 = torch.ones(100)                # 类型1 y data (tensor), shape=(100, )
			
			# 注意 x, y 数据的数据形式是一定要像下面一样 (torch.cat 是在合并数据)
			x = torch.cat((x0, x1), 0).type(torch.FloatTensor)  # FloatTensor = 32-bit floating
			y = torch.cat((y0, y1), ).type(torch.LongTensor)    # LongTensor = 64-bit integer
			
			plt.scatter(x.data.numpy()[:, 0], x.data.numpy()[:, 1], c=y.data.numpy(), s=100, lw=0, cmap='RdYlGn')
			plt.show()
			
			# 画图
			# plt.scatter(x.data.numpy(), y.data.numpy())
			# plt.show()
			
			import torch
			import torch.nn.functional as F     # 激励函数都在这
			
			#搭建正向传递网络方式一:
			class Net(torch.nn.Module):     # 继承 torch 的 Module
			def __init__(self, n_feature, n_hidden, n_output):
			super(Net, self).__init__()     # 继承 __init__ 功能
			self.hidden = torch.nn.Linear(n_feature, n_hidden)   # 隐藏层线性输出
			self.out = torch.nn.Linear(n_hidden, n_output)       # 输出层线性输出
			
			def forward(self, x):
			# 正向传播输入值, 神经网络分析出输出值
			x = F.relu(self.hidden(x))      # 激励函数(隐藏层的线性值)
			x = self.out(x)                 # 输出值, 但是这个不是预测值, 预测值还需要再另外计算
			return x
			
			net = Net(n_feature=2, n_hidden=10, n_output=2) # 几个类别就几个 output
			
			print(net)  # net 的结构
			"""
			Net (
			(hidden): Linear (2 -> 10)
			(out): Linear (10 -> 2)
			)
			"""
			# #搭建正向传递方式二:
			# net2 = torch.nn.Sequential(
			#     torch.nn.Linear(2, 10),
			#     torch.nn.ReLU(),
			#     torch.nn.Linear(10, 2)
			# )
			
			
			# optimizer 是训练的工具
			optimizer = torch.optim.SGD(net.parameters(), lr=0.02)  # 传入 net 的所有参数, 学习率
			# 算误差的时候, 注意真实值!不是! one-hot 形式的, 而是1D Tensor, (batch,)
			# 但是预测值是2D tensor (batch, n_classes)
			loss_func = torch.nn.CrossEntropyLoss()
			
			for t in range(100):
			out = net(x)     # 喂给 net 训练数据 x, 输出分析值
			
			loss = loss_func(out, y)     # 计算两者的误差
			
			optimizer.zero_grad()   # 清空上一步的残余更新参数值
			loss.backward()         # 误差反向传播, 计算参数更新值
			optimizer.step()        # 将参数更新值施加到 net 的 parameters 上
			
			
			
			
			import matplotlib.pyplot as plt
			
			plt.ion()   # 画图
			plt.show()
			
			# for t in range(100):
			#
			#     out = net(x)  # 喂给 net 训练数据 x, 输出分析值
			#
			#     loss = loss_func(out, y)  # 计算两者的误差
			#     loss.backward()
			#     optimizer.step()
			#
			#     # 接着上面来
			#     if t % 2 == 0:
			#         plt.cla()
			#         # 过了一道 softmax 的激励函数后的最大概率才是预测值
			#         prediction = torch.max(F.softmax(out), 1)[1]
			#         pred_y = prediction.data.numpy().squeeze()
			#         target_y = y.data.numpy()
			#         plt.scatter(x.data.numpy()[:, 0], x.data.numpy()[:, 1], c=pred_y, s=100, lw=0, cmap='RdYlGn')
			#         accuracy = sum(pred_y == target_y)/200.  # 预测中有多少和真实值一样
			#         plt.text(1.5, -4, 'Accuracy=%.2f' % accuracy, fontdict={'size': 20, 'color':  'red'})
			#         plt.pause(0.1)
			#
			# plt.ioff()  # 停止画图
			# plt.show()
		\end{python}
	
	\subsection{save and load 参数}
	\begin{python}
		import torch
		torch.manual_seed(1)    # reproducible
		
		# 假数据
		x = torch.unsqueeze(torch.linspace(-1, 1, 100), dim=1)  # x data (tensor), shape=(100, 1)
		y = x.pow(2) + 0.2*torch.rand(x.size())  # noisy y data (tensor), shape=(100, 1)
		
		def save():
		# 建网络
		net1 = torch.nn.Sequential(
		torch.nn.Linear(1, 10),
		torch.nn.ReLU(),
		torch.nn.Linear(10, 1)
		)
		optimizer = torch.optim.SGD(net1.parameters(), lr=0.5)
		loss_func = torch.nn.MSELoss()
		
		# 训练
		for t in range(100):
		prediction = net1(x)
		loss = loss_func(prediction, y)
		
		optimizer.zero_grad()
		loss.backward()
		optimizer.step()
		
		torch.save(net1, 'net.pkl')  # 方式一:保存整个网络
		torch.save(net1.state_dict(), 'net_params.pkl')   # 方式二:只保存网络中的参数 (速度快, 占内存少)
		
		#这种方式将会提取整个神经网络, 网络大的时候可能会比较慢.
		def restore_net():
		# restore entire net1 to net2
		net2 = torch.load('net.pkl')
		prediction = net2(x)
		
		#这种方式将会提取所有的参数, 然后再放到你的新建网络中.
		def restore_params():
		# 新建 net3
		net3 = torch.nn.Sequential(
		torch.nn.Linear(1, 10),
		torch.nn.ReLU(),
		torch.nn.Linear(10, 1)
		)
		
		# 将保存的参数复制到 net3
		net3.load_state_dict(torch.load('net_params.pkl'))
		prediction = net3(x)
	\end{python}

	
	
	
	
%	\begin{listing}[H]
%		\begin{pythoncode}
%			import tensorflow
%			dfjle
%			jfoe
%		\end{pythoncode}	
%	\end{listing}
\end{document}