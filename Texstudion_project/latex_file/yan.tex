% !Mode:: "TeX:UTF-8"
\documentclass{article}
\input{en_preamble.tex}
\input{xecjk_preamble.tex}
\begin{document}
\title{四边形网格生成}
\author{闫世祥}
\date{\chntoday}
\maketitle
\section{提取网格的边}

\begin{listing}[H]
    \begin{pythoncode}
        edge = np.hstack((elements,elements[:, 1:], np.array([elements[:,0]]).T))
        np.reshape(edge,[elements.shape[0]*elements.shape[1], 2], order = "F")
    \end{pythoncode}
\end{listing}

其原理就是把 $element$ 按列拉长, 再把
$elements[:, 1:], np.array([elements[:,0]]).T$ 按列拉长, 然后进行水平拼接, 就
得到的数组每一行都是一个边.

\section{对每组边进行编号}
\begin{listing}[H]
    \begin{pythoncode}
        edge2node,ie=np.unique(np.sort(edge), return_inverse=True,axis = 0)
    \end{pythoncode}
\end{listing}
其中, edge2node 包含所有的边(不重复), ie 是 edge 在 edge2node 中的编号
\section{每个单元每个边在 edge2node 下的编号}
\begin{listing}[H]
    \begin{pythoncode}
        element4edge = np.reshape(ie,[int(len(ie)/4),4], order = "F")
    \end{pythoncode}
\end{listing}
\section{标记需要加密的边}
假设我们已经知道了要加密的单元编号为 marked.
\begin{listing}[H]
    \begin{pythoncode}
        edge4newnode = np.zeros([1,len(edge2node)])
        edge4newnode(element4edge(marked,:)) = 1
    \end{pythoncode}
\end{listing}
现在, edge4newnode 中的元素就是需要加密的边在 edge2node 中的编号.

































\cite{knupp2012introducing}
\bibliographystyle{abbrv}
\bibliography{ref}
\end{document}
