\documentclass{article}
\usepackage{ctex}
\title{git笔记}
\author{Shixiang yan}
\date{\today}

\begin{document}
	\maketitle
	\section{设置公匙}
	ssh-keygen -t rsa
	
	cd ~/.ssh
	
	cat id\_rsa.pub
	\section{添加到github}
	setting
	
	SSH and GPGkeys
	
	New SSH key
	
	测试:ssh -T git@github.com(显示:successfully authenticated)
	\section{全局配置}
	git config --global user.name 'y809777421'
	
	git config --global user.email'shxiangyan@gmail.com'
	\section{push文件}
	
	git init
	
	git pull
	
	git status
	
	git add .

        (git rm --cached <file>删除提交到本地缓存库的文件,还原没有add之前的状态)
        (git checkout -- <file> 取消改变文件操作)
        
        (git reset --hard HEAD 把add的文件转换为没有add的状态)


        git commit -m 'information'

	
	git remote add origin git@github.com:y809777421:xiaoli.git(给git地址添加别名,之后的origin代替git链接)
	
	git push -u origin master(第一次推送需要加上-u)
	
	\section{更改仓库地址}
	git remote rm origin
	
	git remote add origin [url]



        \section{版本回退}
        首先查看版本列表:
        git log
        git lg
        git log --pretty=oneline 一个版本显示一行信息
        git log --oneline 一行信息减少(推荐)
        git reflog

        git reset --hard 哈希值
          三个参数:
          --soft(本地库)绿色
          --mixed(本地库、工作区)红色
          --hard(本地库、暂存区、工作区)

          \section{比较文件的修改情况}
          git diff <file\_name>
          git diff HEAD 比较所有修改的文件

          \section{分支操作}
          查看所拥有的所有分支 类似screen的C-a w操作
          git branch -v

          创建分支
          git branch 分支名 类似C-a c和screen -S session名
          在各分支操作不影响master分支,并且各分支操作后可以合并到master分支

          git push origin 分支名
          
          切换分支
          git checkout 分支名 类似C-a H\_or\_L

          合并分支(合并指的是将在从文件中做的修改加载到主文件上,commit以后才能进行合并)
          首先切换的到被合并的分支 :git checkout master
          其次:git merge 分支名  要合并的分支名

          同时修改同一文件产生冲突.
          <<<<<指示当前分支的内容
          =====

          >>>>>>从另一个分支中merge的内容
          (这些符号需要修改)

          \section{团队合作}
          邀请成员:
          仓库:setting->collaborators->输入邀请人的gitbhu账号->复制链接->被邀请人登录账号访问连接

          pull=feach+marge
          
          (从远程仓库抓取:git fetch origin master
          
          抓取以后本地文件并没有变,
          git checkout origin/master查看在远程master的内容

          git checkout maseter
          
          git merge origin/master 将远程的maser合并到本地)(git pull origin master)


          \section{多团队协作}
          folk->pull requset
          1.folk
          2.克隆到本地
          3.本地修改然后推送
          4.pull request->new pull request->create pull request
          5.主人点pull request->merge pull request(files change查看文件的修改)->comfirm marge
          
          
          
          

          

          
	\begin{thebibliography}{99}
		\bibitem{website1}
		\emph{在Ubuntu18.04系统中向GitHub提交代码}2020.
		Available at \texttt{https://ywnz.com/linuxjc/2401.html}.
		\bibitem{website2}
		\emph{GitHub 系列之「向GitHub 提交代码」}2020.
		Available at \texttt{https://www.jianshu.com/p/d136dee10561}.
		\bibitem{website3}
		\emph{git修改远程仓库地址}2020.
		Available at
		\texttt{https://ddnode.com/2015/04/14/git-modify-remote-responsity-url.html}.
	\end{thebibliography}
	
	
\end{document}
