\section{Results}

\subsection{Conflict of Interest}
When an author or the institution of the author has a relationship, financial or otherwise, with individuals or organizations that could influence the author’s work inappropriately, a conflict of interest may exist. Examples of potential conflicts of interest may include but are not limited to academic, personal, or political relationships; employment; consultancies or honoraria; and finical connections such as stock ownership and funding. Although an author may not feel that there are conflicts, disclosure of relationships and interests that could be viewed by others as conflicts of interest affords a more transparent and prudent process. All authors must disclose any actual or potential conflict of interest. The Proceedings will publish such disclosures if judged to be important to readers.

\subsection{Statement of informed Consent}
Patients have a right to privacy that should not be infringed without informed consent. Identifying information, including patients' names, initials, or hospital numbers, should not be published in written descriptions, photographs, and pedigrees unless the information is essential for scientific purposes and the patient (or parent or guardian) gives written informed consent for publication. Informed consent for this purpose requires that a patient who is identifiable be shown the manuscript to be published. Authors should identify Individuals who provide writing assistance and disclose the funding source for this assistance. 
Identifying details should be omitted if they are not essential. Complete anonymity is difficult to achieve, however, and informed consent should be obtained if there is any doubt. For example, masking the eye region in photographs of patients is inadequate protection of anonymity. If identifying characteristics are altered to protect anonymity, such as in genetic pedigrees, authors should provide assurance that alterations do not distort scientific meaning and editors should so note.

\subsection{Statement of human and animal rights}
When reporting experiments on human subjects, authors should indicate whether the procedures followed were in accordance with the ethical standards of the responsible committee on human experimentation (institutional and national) and with the Helsinki Declaration of 1975, as revised in 2000 and 2008. If doubt exists whether the research was conducted in accordance with the Helsinki Declaration, the authors must explain the rationale for their approach, and demonstrate that the institutional review body explicitly approved the doubtful aspects of the study. When reporting experiments on animals, authors should be asked to indicate whether the institutional and national guide for the care and use of laboratory animals was followed.
