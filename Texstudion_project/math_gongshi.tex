\documentclass{article}
\usepackage{ctex}
\usepackage{amsmath}
\begin{document}
	\section{简介}
	\LaTeXe{}将排版内容分为文本模式和数学模式。文本模式用于普通文本排版,数学模式用于数学公式排版
	
	
	\section{行内公式}
	
	\subsection{美元符号}
	交换律是$a+b=b+a$,如$1+2=2+1=3$。
	
	\subsection{小括号}
	交换律是\(a+b=b+a\),如\(1+2=2+1=3\)。
	
	\subsection{math环境}
	交换律是\begin{math}
		a+b=b+a
	\end{math}
	如\begin{math}
		1+2=2+1=3
	\end{math}。
	
	
	\section{上下标}
	\subsection{上标}
	$3x^2-x+2=0$
	
	$3x^{20}+2=0$%当指数是多位数时要用大括号进行分组
	
	\subsection{下标}
	$a_0,a_{100}$
	
	\section{希腊字母}
	$\alpha$
	$\beta$
	$\gamma$
	$\epsilon$
	$\omega$
	
	$\Gamma$
	$\Delta$
	$\Theta$
	$\Pi$
	$\Omega$
	
	$\alpha^3+\beta^2+\gamma=0$
	
	\section{数学函数}
	$\log$
	$\sin$
	$\cos$
	$\arcsin$
	$\arccos$
	$\ln$
	
	$\sin^2 x+\cosh^2 y=1$
	
	$y=\arccos x$
	
	$y=\sin^{-1} x$
	
	$y=\ln x$
	
	$\sqrt{2}$
	
	$\sqrt{x^2+y^2}$
	
	$\sqrt{2+\sqrt{20}}$
	
	$\sqrt[4]{x}$
	
	\section{分式}
	大约是原体积的$3/4$。
	大约是原体积的$\frac{3}{4}$
	
	$\frac{1}{1+\frac{1}{x}}$
	
	\section{行间公式}
	\subsection{美元符号}
	%方式一:使公式占一行并居中,应用两个美元。
	交换律:$$a+b=b+a$$

	\subsection{中括号}
	%方式二:
	交换律是\[a+b=b+a\]
	
	\subsection{displaymath环境}
	%方式三:
	交换律:
	\begin{displaymath}
	a+b=b+a
	\end{displaymath}
	
	\subsection{自动编号公式环境}
	交换律是:
	\begin{equation}
		a+b=b+a
	\end{equation}
	%交叉引用
	结合律如式\ref{key}所示:
	\begin{equation}
	a+b=b+a\label{key}
	\end{equation}
	
	\subsection{不编号equation*公式环境}
	\begin{equation*}
		a+b=b+a
	\end{equation*}
\end{document}