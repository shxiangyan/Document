%仅有10、11、12pt
\documentclass[12pt]{article}
\usepackage{ctex}
%定义字体格式命令,定义命令
\newcommand{\myfont}{\textit{\textbf{\textsf{Fancy Text}}}}
%正文区(文稿区)
\begin{document}
	%方式一:字体族设置(罗马字体、无衬线字体、打字机字体)
	\textrm{罗马字体Roman:Family}
	%无衬线字体
	
	\textsf{无衬线字体:Sans Serif Family}
	%打字机字体
	
	\texttt{打字机字体:Typewriter Family}
	
	%字体形状(直立、斜体、伪斜体、小型大写)
	\textup{直立:Upright Shape}
	
	\textit{斜体:Italic Shape}
	
	\textsl{伪斜体:Slanted Shape}
	
	\textsc{小型大写:Small Caps Shape}
	
	%方式二:声明后续字体,可用大括号进行分组。知道若无大括号知道遇到下一个字体声明才结束当前字体声明,同样可以用大括号分组限定字体声明范围。
	\rmfamily {Roman} Family
	
	\sffamily Sans Serif Family
	
	\ttfamily Typewriter Family
	
	{\upshape Upright Shape}
	
	{\itshape Italic Shape}
	
	{\slshape Slanted Shape}
	
	{\scshape Small Caps Shape}
	%中文字体,帮助文档(texdoc ctex)
	
	{\songti 宋体}\quad1
	
	{\heiti 黑体}\quad2
	
	{\fangsong 仿宋}\quad3
	
	{\kaishu 楷书}\quad4
	
	中文字体的\textbf{粗体}与\textit{斜体} 
	%字体大小
	{\tiny Hello}\\
	{\scriptsize Hello}\\
    {\normalsize Hello}\\%normalsiz大小的设置在初始化文档类时
    {\small Hello}\\
    {\normalsize Hello}\\
    {\large Hello}\\
    {\Large Hello}\\
    {\LARGE Hello}\\
    {\huge hello}\\
    {\Huge hello}\\
	
    %中文字号设置命令
    \zihao{-0}你好!\\
    \zihao{5}你好!\\
    \myfont
\end{document}