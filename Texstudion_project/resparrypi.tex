\documentclass{article}
\usepackage{ctex}

\title{树莓派笔记}
\author{Shixiang Yan}
\date{\today}

\begin{document}
	\maketitle
	\tableofcontents
	\section{软路由}
	\subsection{刷路由系统}
	材料:
	OpenWRT-for-RaspberryPi4-by-Tony.img
	USBWriter.exe
	\subsection{设置树莓派网关}
	接口$->$上级路由器的ip作为网关,自设树莓派ip,子网掩码以及DNS$->$确定。
	\subsection{添加订阅}
	芝麻开门$->$shadowsocks$->$添加订阅$->$更新$->$启动订阅
	\subsection{通过无线热点连接}
	更改为静态ip$->$网关为树莓派ip
	
	ubutu18.04更改ip:首先,vi /etc/network/interfaces;然后,添加:
	
		auto eth0
		iface eth0 inet static
		address 192.168.8.100    
		netmask 255.255.255.0
		gateway 192.168.8.2

		最后:sudo /etc/init.d/networking restart
	
	设置DNS:/etc/systemd/resolved.conf,[Resolve],DNS=119.29.29.29,systemctl restart systemd-resolved.service
	
	\subsection{备注}
	上级路由ip192.168.1.1,腾达路由ip192.168.1.1,树莓派路由ip192.168.1.2
	\section{机器人}
\end{document}